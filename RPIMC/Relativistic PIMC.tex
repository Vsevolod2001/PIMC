\documentclass{article}
\usepackage{graphicx}
\usepackage{amsmath}

\title{Relativistic Path Integral Monte-Carlo}
\date{December 2023}

\begin{document}

\maketitle

\section{Density matrix and energy}
	Hamiltonian
	\begin{equation}
		H = \sqrt{p ^ 2 + m ^ 2} + V(q).
		\label{Main Hamiltonian}
	\end{equation}
	Density matrix at small $\tau$
	\begin{equation}
		\rho (q '', q'; \tau) = \langle q '' | e ^ {- \tau H} | q ' \rangle.
	\end{equation}
	Density matrix in coordinate representation
	\begin{align}
		\nonumber
		\rho (q '', q'; \tau) =& \Biggl ( \frac{m \tau}{\pi \sqrt{\tau ^ 2 + (q '' - q ') ^ 2}} \Biggr ) ^ {(d + 1) / 2} \cdot \\
    	\cdot& \frac{K _ {(d + 1) / 2} (m \sqrt{\tau ^ 2 + (q '' - q ') ^ 2})}{(2 \tau) ^ {(d - 1) / 2}} e ^ {- \tau V (q ')},
    	\label{Density Matrix General}
	\end{align}
	where $d = 1, 2, 3$.

	Markov chain limit probability density for $d = 1$
	\begin{equation}
 		\pi (q _ i) = \frac{K _ 1 \Bigl [ m \sqrt{\tau ^ 2 + (q _ i - q _ {i - 1}) ^ 2} \Bigr ] K _ 1 \Bigl [ m \sqrt{\tau ^ 2 + (q _ {i + 1} - q _ i) ^ 2} \Bigr ]}{\sqrt{\tau ^ 2 + (q _ i - q _ {i - 1}) ^ 2} \sqrt{\tau ^ 2 + (q _ {i + 1} - q _ i) ^ 2}} e ^ {-\tau V(q _ i)}.
 		\label{MCLPD d = 1}
	\end{equation}
	Average value of kinetic energy for $d = 1$
	\begin{equation}
		\langle \sqrt{p ^ 2 + m ^ 2} \rangle = \Bigl \langle \frac{m \tau}{\sqrt{\tau ^ 2 + (\Delta q) ^ 2}} \frac{K _ 0 (m \sqrt{\tau ^ 2 + (\Delta q) ^ 2})}{K _ 1 (m \sqrt{\tau ^ 2 + (\Delta q) ^ 2})} + \frac{\tau ^ 2 - (\Delta q) ^ 2}{\tau (\tau ^ 2 + (\Delta q) ^ 2)} \Bigr \rangle.
		\label{Kinetic Energy d = 1}
	\end{equation}

	Density matrix in coordinate representation for $d = 2$
	\begin{align}
		\nonumber
		\rho (q '', q '; \tau) =& \frac{m ^ 3 \tau}{2 \pi} \cdot \frac{1 + m \sqrt{\tau ^ 2 + (q '' - q ') ^ 2}}{\bigl ( m \sqrt{\tau ^ 2 + (q '' - q ') ^ 2} \bigr ) ^ 3} \cdot \\
		\cdot& \exp \Bigl [ - m \sqrt{\tau ^ 2 + (q '' - q ') ^ 2} \Bigr ] e ^ {- \tau V (q ')}.
		\label{Density Matrix d = 2}
	\end{align}
	Markov chain limit probability density for $d = 2$
	\begin{align}
		\pi (q _ i) =& \frac{1 + m \sqrt{\tau ^ 2 + (q _ i - q _ {i - 1}) ^ 2}}{\bigl ( m \sqrt{\tau ^ 2 + (q _ i - q _ {i - 1}) ^ 2} \bigr ) ^ 3} \cdot \frac{1 + m \sqrt{\tau ^ 2 + (q _ {i + 1} - q _ i) ^ 2}}{\bigl ( m \sqrt{\tau ^ 2 + (q _ {i + 1} - q _ i) ^ 2} \bigr ) ^ 3} \cdot \\
		\cdot& \exp \Bigl [ - m \sqrt{\tau ^ 2 + (q _ i - q _ {i - 1}) ^ 2} - m \sqrt{\tau ^ 2 + (q _ {i + 1} - q _ i) ^ 2} \Bigr ] e ^ {- \tau V (q _ i)}.
		\label{MCLPD d = 2}
	\end{align}
	Average value of kinetic energy for $d = 2$
	\begin{equation}
		\langle \sqrt{p ^ 2 + m ^ 2} \rangle = \dots
		\label{Kinetic Energy d = 2}
	\end{equation}

\section{RPIMC calculations for oscillator model in two dimensions}
	Non-relativistic (NR) and ultra-relativistic (UR) limit energy
	\begin{equation}
		E _ {NR} = \dots ,
		\label{Non-relativistic energy d = 2}
	\end{equation}
	\begin{equation}
		E _ {UR} = \dots .
		\label{Ultra-relativistic energy d = 2}
	\end{equation}


\section{Neural network calculations}
	Oscillator potential
	\begin{equation}
		V (q) = \frac{1}{2} m \omega ^ 2 q ^ 2.
	\end{equation}
	Let's choose parameters values
	\begin{align}
		\nonumber
		& m = 100, \\
		\nonumber
		& \omega = 1, \\
		\nonumber
		& N _ t = 10, \\
		\nonumber
		& \tau = e / 10 \approx 0.2718 \dots .
	\end{align}

\end{document}
