\documentclass[12pt,a4paper,oneside]{article}
\usepackage[utf8]{inputenc}
\usepackage[russian]{babel}
\usepackage[OT1]{fontenc}
\usepackage{amsmath}
\usepackage{amsfonts}
\usepackage{amssymb}
\usepackage{graphicx}
\usepackage[left=2cm,right=2cm,top=2cm,bottom=2cm]{geometry}
\usepackage{multirow}
\usepackage{lettrine}

\usepackage[unicode, pdftex]{hyperref}

\usepackage{xcolor}
\usepackage{hyperref}

\usepackage{ragged2e}
 
\title{Обезразмеривание релятивистского осциллятора}
\author{Сальников Дмитрий}

\begin{document}
\maketitle

Я считаю, что лучше сразу работать в естественной системе $\hbar = c = 1$, так будет удобнее. Так же в технической реализации, на мой взгляд, стоит сразу вычесть массу из полной энергии, чтобы в нерелятивистком пределе получался стандартный гамильтониан без большой постоянной массовой добавки. Сева утверждал, что с этим могут возникнуть проблемы, которые возникают с гамильтонианом Дирака, но я думаю, что здесь их быть не должно, поскольку дисперсионное соотношение для такого гамильтониана явно не включают отрицательные энергии, как это происходит в теории с гамильтонианом Дирака.

В таком случае гамильтониан и коммутационные соотношения в произвольном числе измерений ($p \equiv |{\bf p}|$, $q \equiv |{\bf q}|$)  имеют вид :

\begin{equation}\label{general}
H = \sqrt{p^2 + m^2} - m + \dfrac{m\omega^2 q^2}{2},
\end{equation}
\begin{equation}
[q_i, p_j] = i\delta_{ij}.
\end{equation}


Если обезразмеривать величины единым образом для любых масс и частот, могут быть проблемы в одном из предельных случаев. Поэтому разумно будет применить два разных варианта для случаев $m < \omega$ и $m > \omega$.

В случае $m > \omega$ в нерелятивистком пределе гамильтониан (\ref{general}) будет иметь стандартный вид:

\begin{equation}
H = \dfrac{p^2}{2m} + \dfrac{m\omega^2 q^2}{2}.
\end{equation}

Переход к безразмерным переменным происходит стандартным образом:

\begin{equation}
p \to \sqrt{m\omega} \, p, \quad q \to \frac{q}{\sqrt{m\omega}}, \quad H \to \omega H.
\end{equation}

Таким образом, безразмерный гамильтаниан будет иметь вид:

\begin{equation}\label{nonrelativistic}
H =  \dfrac{p^2}{2} + \dfrac{q^2}{2}.
\end{equation}

В случае $m > \omega$ перейдём к безразмерным переменным в гамильтониан (\ref{general}) точно так же

\begin{equation}
\omega H = \sqrt{m\omega p^2 + m^2} - m + \dfrac{\omega q^2}{2},
\end{equation}

\begin{equation}
H = \sqrt{\left(\frac{m}{\omega}\right) p^2 + \left(\frac{m}{\omega}\right)^2} -  \dfrac{m}{\omega} + \dfrac{q^2}{2}.
\end{equation}

Дальше воспользуемся стандартным приёмом:

\begin{equation}
H = \left[ \sqrt{\left(\frac{m}{\omega}\right) p^2 + \left(\frac{m}{\omega}\right)^2} -  \dfrac{m}{\omega} \right] \times \dfrac{ \left[ \sqrt{\left(\frac{m}{\omega}\right) p^2 + \left(\frac{m}{\omega}\right)^2} +  \frac{m}{\omega} \right]}{ \left[ \sqrt{\left(\frac{m}{\omega}\right) p^2 + \left(\frac{m}{\omega}\right)^2} +  \frac{m}{\omega} \right]} + \dfrac{q^2}{2},
\end{equation}

\begin{equation}
H =  \dfrac{ \left(\frac{m}{\omega}\right) p^2 }{ \left[ \sqrt{\left(\frac{m}{\omega}\right) p^2 + \left(\frac{m}{\omega}\right)^2} +  \frac{m}{\omega} \right]} + \dfrac{q^2}{2},
\end{equation}

\begin{equation}\label{largemass}
H =  \dfrac{ p^2 }{ 1 + \sqrt{1 + \left(\frac{\omega}{m}\right) p^2}} + \dfrac{q^2}{2}.
\end{equation}

Таким образом, при $m > \omega$ безразмерный параметр $\frac{\omega}{m}$ всегда меньше единицы, и при $m \gg \omega$ гамильтониан переходит в (\ref{nonrelativistic}). 

Теперь рассмотрим ультрарелятивисткий случай:

\begin{equation}
H = p - m + \dfrac{m\omega^2 q^2}{2}.
\end{equation}

Обезразмеривание в таком пределе имеет вид:

\begin{equation}
p \to (m\omega^2)^{1/3}\, p, \quad q \to \frac{q}{(m\omega^2)^{1/3}}, \quad H \to (m\omega^2)^{1/3}  H,
\end{equation}

\begin{equation}
(m\omega^2)^{1/3} H = (m\omega^2)^{1/3} p - m + \dfrac{m\omega^2 q^2}{2(m\omega^2)^{2/3}},
\end{equation}

\begin{equation}\label{relativistic}
H = p - \left(\dfrac{m}{\omega}\right)^{2/3} + \dfrac{q^2}{2}.
\end{equation}

Поэтому в случае $m < \omega$ имеет смысл обезразмеривать гамильтониан (\ref{general}) так же:

\begin{equation}
(m\omega^2)^{1/3} H = \sqrt{(m\omega^2)^{2/3} p^2 + m^2} - m + \dfrac{m\omega^2 q^2}{2 (m\omega^2)^{2/3}},
\end{equation}

\begin{equation}
 H = \sqrt{p^2 + \left(\frac{m}{\omega} \right)^{4/3}} - \left(\frac{m}{\omega} \right)^{2/3}  + \dfrac{q^2}{2},
\end{equation}

Таким образом, при $m < \omega$ безразмерный параметр $\frac{m}{\omega}$ всегда меньше единицы, и при $m \ll \omega$ гамильтониан переходит в (\ref{relativistic}). 


\

Итог:

\

1. $m > \omega$

\begin{equation}
p \to \sqrt{m\omega} \, p, \quad q \to \frac{q}{\sqrt{m\omega}}, \quad H \to \omega H.
\end{equation}

\begin{equation}\label{largemass}
H =  \dfrac{ p^2 }{ 1 + \sqrt{1 + \left(\frac{\omega}{m}\right) p^2}} + \dfrac{q^2}{2}
\end{equation}

если не вычитать из исходного гамильтониана  массу:

\begin{equation}\label{largemass}
H =  \dfrac{ p^2 }{ 1 + \sqrt{1 + \left(\frac{\omega}{m}\right) p^2}} + \dfrac{q^2}{2} + \dfrac{m}{\omega}
\end{equation}

2. $m < \omega$

\begin{equation}
p \to (m\omega^2)^{1/3}\, p, \quad q \to \frac{q}{(m\omega^2)^{1/3}}, \quad H \to (m\omega^2)^{1/3}  H,
\end{equation}

\begin{equation}
 H = \sqrt{p^2 + \left(\frac{m}{\omega} \right)^{4/3}} - \left(\frac{m}{\omega} \right)^{2/3}  + \dfrac{q^2}{2},
\end{equation}

если не вычитать из исходного гамильтониана  массу:

\begin{equation}
 H = \sqrt{p^2 + \left(\frac{m}{\omega} \right)^{4/3}} + \dfrac{q^2}{2},
\end{equation}

3. $m = \omega$, оба варианта дадут

\begin{equation}
 H = \sqrt{p^2 + 1} - 1 + \dfrac{q^2}{2},
\end{equation}

если не вычитать из исходного гамильтониана  массу:

\begin{equation}
 H = \sqrt{p^2 + 1} + \dfrac{q^2}{2},
\end{equation}

\end{document}






